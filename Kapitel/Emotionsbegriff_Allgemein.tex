\chapter{Begriffserklärung Emotionen}

Emotionen sind schon seit Jahrtausenden von Jahren ein wichtiger Gegenstand der Forschung. Dabei wurde die Definition bzw. Interpretation von Emotionen und deren dazugehörigen emotionalen Begriffe, wie Stimmung, Affekt oder Gefühl, entsprechend der gerade vorherrschenden Strömung angepasst oder erweitert. Bis heute allerdings, konnten sich Psychologen, Philosophen und Biologen weder auf eine klare Definition des Begriffs Emotion noch auf eine klare Abgrenzung zu den anderen emotionalen Begriffen einigen.

Die fehlende begriffliche Abgrenzung und das damit verbundene fehlende Konzept, was eine Emotion ist, führt zu diversen Problemen in der Erforschung des Phänomens Emotion. Der aktuelle Status Quo besteht darin, dass je nach Disziplin, Untersuchungsmethode und historischen Hintergrund diverse Arbeitshypothesen, welche zum einen umfasst was eine Emotion ist und zum anderen wie sie entsteht, entworfen und mit entsprechenden Messmethoden und dazugehöriger Argumentation nachgewiesen werden. Dies führte zu teils nicht wiederholbaren oder widersprüchlichen Forschungsergebnissen, welche die Definition eines universellen Emotionsbegriffes fast unmöglich macht.

Im folgenden Kapitel werden die verschiedenen Arbeitshypothesen in verschiedenen Disziplinen und die darin entwickelten wichtigsten und heute noch bedeutsamsten Theorien von Emotionen beschrieben. Es wird des weiteren eine Interpretation von Emotionen vorgestellt, welche eine Arbeitsdefinition von Emotionen bereitstellt, die sowohl philosophisch, psychologisch, biologisch und auch computional plausibel erscheint allerdings noch sehr vage formuliert ist. Aus diesem Grund erfolgt im abschließenden Unterkapitel die für diese Arbeit verwendete Arbeitshypothese der Emotion, welche hinsichtlich zugrunde liegender Theorien aus biologischer, psychologischer und philosophischer Theorien analysiert und anhand eigens durchgeführter Experimente belegt werden soll.  

% Hierbei spielen mehrere Aspekte der Psychologie (Sokolowsky, 2002),
% Robotik (Bartneck and Fourlizzi, 2004; Tapus et al., 2007) und Künstlichen Intelligenz (Salichs and
% Malfaz, 2012; Bartneck et al., 2008) eine große Rolle.

\section{Wortherkunft}
Das Wort Emotion bedeutet Gefühls- oder Gemütsbewegung und stammt aus dem lateinischen von \textit{emotio} für \"{}das Fortbewegen\"{} und dem dazugehörigen Verb \textit{emovere} für \"{}„herausbewegen, –schaffen, um und um bewegen, erschüttern, aufwühlen\"{} ab. \textit{Emovere} stetz sich aus den zwei Wortbausteinen \textit{e-} für \"{}aus, heraus\"{} und \textit{movere} für \"{}bewegen\"{} zusammen.

\section{Begriffe für Emotionen}
\"{}páthos\"{} - Aristoteles, Chrysipp\\
affectus - Seneca\\
perturbationes (passiones, affectus oder affectiones) - Augustinus\\
passio - Thomas von Aquin\\
passions - Rene Descartes\\
affectus, passio - Spinoza\\








