\chapter{Kurzbeschreibung}
Ziel der Arbeit ist die Entwicklung eines Emotions und Workloadmodells, welches zum einen die Emotionen von
Menschen berechnen und vorhersagen kann und zum anderen, eingesetzt in Robotern, diesen
ermöglichen soll, ein auf den Menschen zugeschnittenes individuelles und unterstützendes Verhalten zu
generieren (Fong et al., 2003). Hierbei spielen mehrere Aspekte der Psychologie (Sokolowsky, 2002),
Robotik (Bartneck and Fourlizzi, 2004; Tapus et al., 2007) und Künstlichen Intelligenz (Salichs and
Malfaz, 2012; Bartneck et al., 2008) eine große Rolle.
Für die Simulation menschlicher Emotionen existieren in der Literatur viele Emotionsmodelle, die das
emotionale Verhalten von Menschen erklären bzw. simulieren soll (Marsella and Gratch, 2010). Die
Modelle sind unterschiedlich komplex und verfolgen unterschiedliche theoretische Ansätze der
Psychologie (Marsella and Gratch, 2014). Allerdings sind wenige dieser Modelle mit realen Messdaten
aus psychologischen Experimenten validiert worden (Marinier et al., 2013). Das in der Dissertation
entwickelte Emotionsmodell soll sowohl den Anspruch der theoretischen Fundiertheit in der
Psychologie als auch deren funktionale Korrektheit mit Hilfe realer Experimentdaten nachweisen
können.
Es wurden unterschiedliche Ansätze für den Transfer von psychologischen Ergebnissen in ein Modell
angewendet und untersucht. Die jeweiligen Ergebnisse mit ihren Vor- und Nachteilen werden in dieser
Arbeit gegenübergestellt.
Der erste Ansatz wurde in Müller und Truschzinski (2013) beschrieben und beinhaltet die
Implementierung eines bestehenden psychologischen Modells innerhalb einer Menschsimulation. Die
auf dieser Grundlage entwickelte Berechnungsvorschrift der emotionalen Erregung des Menschen
während der Arbeit wurde anschließend als Rewardfunktion innerhalb eine Reinforcementlearners
implementiert (Truschzinski et al., 2014). Anschließend wurden die Ergebnisse mit Hilfe eines erneuten
psychologischen Experimentes validiert (Müller und Truschzinski, 2015). Dabei haben sich einige
Probleme ergeben. Zum einen stellte es sich heraus, dass ein Computer nicht nur die
Berechnungsvorschrift sondern auch schlicht weg Zahlen (Startparameter, Anpassungsparameter), die
innerhalb der Studienveröffentlichungen nicht oder nur am Rande beachtet werden, benötigt. Außerdem
ist die Validierung eines solchen Systems immer schwierig, da der konkrete Anwendungsfall ein
anderer ist als der, der zur Modellentwicklung beigetragen hat.
Aus diesem Grund wurde im zweiten Teil der Arbeit ein Ansatz gewählt, der den Mensch als System
betrachtet und die Emotionsfunktion mit Hilfe von Prozessmodellen identifiziert (Pfeiffer et al., 2015).
Hierbei stellte sich heraus, dass die Wahl der aufzunehmenden Daten (Hautleitfähigkeit, Pupillengröße
und Fragebögen) sowie deren Einteilung in objektive und subjektive Daten essentiell war. So gibt es
Unterschiede zwischen subjektiv empfundener Belastung, der in Fragebögen postuliert wird und
objektiv messbarer (kognitiver) Belastung, gemessen anhand der Pupillengröße.
Der in der Arbeit beschriebene zweite Ansatz bietet die Möglichkeit nicht nur bestehende Studien
anhand der mit unterschiedlichen Sensoren gemessenen Daten zu untersuchen sondern zugleich
Vorhersagen zu formulieren, die in weiteren Experimenten weiterführend untersucht werden können.
Aus diesem Grund liefert die Arbeit einen wichtigen Beitrag, die bisher sehr große Lücke zwischen
psychologischer Forschung und der computionalen Modellierung zu schließen, da Daten direkt von der
psychologischen Untersuchung in die Modellbildung integriert und validiert werden können.
