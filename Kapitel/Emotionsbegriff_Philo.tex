\section{Erste Betrachtungen von Emotionen in der Philosophie}
In der Philosophie, als \"{}Wissenschaft von der Erkenntnis des Sinn des Lebens, der Welt und der Stellung des Menschen in der Welt\"{} wurde die Rolle der Emotionen mal als mehr oder weniger bedeutend oder unbedeutend interpretiert, eine zentrale Rolle im menschlichen Sein wurde allerdings lange verneint \cite{robert_c._solomon_philosophy_2008}. Da aber in der Philosophie Emotionen deren Rolle, Entstehung und Bedeutung für das menschliche Leben vielseitig diskutiert wurde und die Philosophie aufgrund der puren Verwendung von Sprache und Logik die größtmögliche Freiheit zur Formulierung einer Theorie besitzt, sollen hier die wichtigen Theorien zu Emotionen präsentiert werden.\\
%
%
Schon \textbf{Aristoteles} (*384 v. Chr. - 322 v. Chr.) beschrieb mit dem Begriff  \"{}passio\"{} Emotionen, dabei bezog er sich in seiner Abhandlung  \"{}Rhetoric\"{} auf Ärger, Gelassenheit, Freundlichkeit/ Feindlichkeit, Angst, Scham/ Schamlosigkeit, Güte,  Mitleid, Empörung, Neid und Wetteifer. Wobei er  \"{}passio\"{} generell als  \"{}diese Gefühle, die Männer verändern indem sie ihr Urteilsvermögen beeinflussen und die meist mit Vergnügen und Schmerz verbunden sind\"{} \cite{aristotle_categories_1938} S.3190ff definierte. Er analysierte alle aufgeführte Emotionen hinsichtlich drei Fragen:
\begin{enumerate}
\item In welcher geistigen Verfassung befindet sich der Mensch während der gefühlten Emotion?
\item Welche Personen rufen diese Emotionen hervor und auf wen beziehen sie sich?
\item Mit welcher Begründung fühlt diese Person die Emotion bezogen auf den Menschen?
\end{enumerate}
Er stellte fest, dass Emotionen nur geweckt werden, wenn alle drei Fragen beantwortet bzw. eine Ursache haben.\\
Allerdings bezog Aristoteles Emotionen eher in einen gesellschaftlichen bzw. ethischen Kontext \cite{robert_c._solomon_philosophy_2008}, so war Ärger für ihn von Interesse, weil es eine natürliche (moralische) Reaktion auf eine Straftat war, welche durch Motive und Rhetorik kultiviert und provoziert werden konnten. Aristoteles sah also Emotionen als zentral und essentiell für ein gutes (Zusammen-)Leben von Menschen \cite{robert_c._solomon_philosophy_2008} an.\\
%
%
Auch im Stoizismus (300 v.Chr. bis 200 n.Chr), eine philosophische Strömung, welche später entscheidend die christliche Ethik beeinflusst hat, wurden Emotionen analysiert. Allerdings wurde innerhalb dieser Strömung Pflichtbewusstsein, Leidenschaftslosigkeit, Gleichmut und tugendhaftes Leben in Übereinstimmung mit der Vernunft und der Natur propagiert. Emotionen bzw. in diesem Kontext auch Affekt genannt hinderten Menschen, laut den im Stoizismus verbreiteten Doktrin, daran ein solches tugendhaftes Leben zu führen. Der Stoiker \textbf{Chrysipp} (281/76 v.Chr-208/4 v. Chr.) benannte dabei vier Basis-Affekte: Trauer, Lust, Furcht und Begierde. 
Wobei er diese Affekte hinsichtlich ihrer Wertung, das heißt hinsichtlich der Frage, ob etwas gut oder schlecht ist, und der zeitlichen Ausrichtung, das heißt hinsichtlich der Frage ob sich der Affekt auf etwas zukünftiges oder gegenwärtiges bezieht, \cite{richard_sorabji_emotion_2000} analysierte. So empfindet man Trauer bezüglich einem gegenwärtiges Leid, Lust bezüglich einem gegenwärtigen Vorteil, Furcht bezüglich etwas, dass zukünftiges Leid hervorrufen kann, Begierde bezüglich etwas das zukünftige eintretende Vorteile verspricht. 
Andere Affekte sind für ihn Unterkategorien dieser vier Basis-Affekte. In seiner Interpretation sind Emotionen Bewertungen der Welt und der eigenen Situation in der Welt und die aus dieser Bewertung resultierende Reaktion \cite{richard_sorabji_emotion_2000}. Eine solche Entscheidung die allein aus einer Emotion heraus getroffen wurde berücksichtigt aber laut Chrysipp nicht alle notwendigen Informationen und Zusammenhänge, aus diesem Grund ist sie eine fehlgeleiteten Entscheidung \cite{richard_sorabji_emotion_2000}.\\
%
%
Der Stoiker \textbf{Seneca} (1 - 65 n. Chr.) unterteilte eine solch emotionale Entscheidung in mehrere Phasen auf. Seiner Meinung nach manifestierte sich nach einer Entscheidung zuerst eine unfreiwillige körperliche Reaktion ("Erster Andrang"), welche zum Beispiel Zittern oder Blässe darstellte, darauf erfolgt die eigentliche vom Willen erlaubte affektive Reaktion, welche anschließend von einer gänzlich unkontrollierten Reaktion gefolgt wird. Das heißt,  eine Emotion sich erst unkontrolliert aufbaut (wobei er diesen Teil vom eigentlichen Affekt ausgeschlossen hat) und wird, wenn der Wille es zu lässt, immer stärker - bis hin zum Exzess. Deshalb ist es das beste, das Eindringen von Emotionen in die Seelen zu verhindern \cite{motto_additional_2009}.\\
Generell wurden Emotionen im Stoizismus als Laster oder fehlgeleitete Entscheidungen des Lebens betrachtet \cite{robert_c._solomon_philosophy_2008}, welche therapiert und durch kognitive Wertungen ersetzt werden müssen, um zu den angestrebten Zustand der Seelenruhe (Apathie) zu gelangen \cite{schafer_passiones_2013}.\\
%
%
Dem entgegenstehend gab es die philosophische Strömung des Epikureismus (307 v. Chr. - 300 n. Chr), welcher im Gegensatz zum Stoizismus nicht davon ausgeht, dass man sich von Gefühlen komplett befreien könnte, sondern dass es unvermeidbare Wertungen gibt, die sich nicht von der Vernunft (komplett) verhindern lassen. Diese unvermeidbaren Wertungen beinhalten dabei ein Streben nach Lust und ein Verhindern von Unlust. \textbf{Epikur}, als ein wichtiger Vertreter des Epikurismus, unterschied dabei natürliche Bedürfnisse, die einen Mangel im Grundzustand des Menschen beheben und den Normalzustand wieder herstellen, und unnatürliche Bedürfnisse, die über die Behebung des Mangels hinausführen. Demzufolge ging er in seiner Theorie davon aus, dass der Mensch seine Seelenruhe nur finden würde, wenn er seine natürlichen Bedürfnisse befriedigen würde, da er sonst von Schmerzen und Unlust geplagt würde und dass Ausschweifungen und Luxus bzw. ein lustvolles Leben vermieden werden sollte. Da nur "nüchternes Rechnen der Vernunft [...] die Gründe allen Wählens und Meidens erforscht und das die Wahnvorstellung vertreibt, derentwegen größte Aufregung die Seele ergreift\"{} \cite{hossenfelder_geschichte_2017} zu der angestrebten Seelenruhe führen würde.\\
%
%
Von \textbf{Augustinus} (354-430 n. Chr.) wurden die Theorien der Epikurer und Stoiker aufgegriffen und gleichermaßen kritisiert. Er erschuf anhand dieser Kritik eine dem christlichen Glauben entsprechende Lehre, in welcher Emotionen überwiegend ethisch reflektiert werden. Für Augustinus gehörten Affekte, er beschrieb, konkret Begierde, Furcht, Lust und Kummer, zur Natur des Menschen denen man nicht entfliehen kann. Allerdings können diese sowohl positiv als auch negativ wirken. So kann ein Affekt, welcher aus heiliger Liebe und guten Willen entstanden ist, der sich in vernünftigen Bahnen hält und dort auftritt wo er angebracht ist keine \"{}sündige Leidenschaft\"{} sein. Er interpretiert diese positiven Emotionen hinsichtlich des christlichen Glaubens, so sind Gott bezogene Affekte, wie Nächstenliebe, Gottes Liebe und das Begehren des Ewigen Lebens positiv besetzt und selbstbezogene Affekte wie Ärger, Stolz und Lust negativ besetzt. Entsprechenden dem christlichen Glauben vertrat auch Augustus die Meinung, dass (natürliche) Emotionen therapiert und durch die Vernunft gezügelt werden müssen \cite{schafer_passiones_2013}.\\
%
%
Thomas von Aquin (1225-1274) beschreibt Emotionen als \"{}gewisse Bewegungen des nichtrationalen Strebens\"{} \cite{forschner_thomas_2006}. Das heißt der (passive) Mensch wird von einer sinnlichen Wahrnehmung, welche gut oder schlecht sein kann, angezogen oder abgestoßen. Diese Emotion geht nach Thomas von Aquin mit einer körperlichen Veränderung tendenziell zum Schlechten einher  \cite{catherine_passion_2009}. Mit Hilfe der von ihm entwickelten Emotionsdefinition unterteilt er 11 verschiedene Grundemotionen, welche anhand ihres Objektes, welches von dem Empfindenden als gut oder schlecht interpretiert wird und dessen Erreichung mit verschiedenen Schwierigkeiten verbunden sein kann. Diese Schwierigkeiten führen zu unterschiedlichen Verhalten zum Beispiel zu einem \"{}aufbegehrenden oder kampffähigen Strebevermögen\"{} \cite{catherine_passion_2009}. Thomas von Aquin definierte in seinen Schriften: Liebe und Hass, Begehren und Meiden, Freude und Trauer sowie Hoffnung und Verzweiflung, Furcht und Wagemut und Zorn, als die 11 Grundemotionen.
Diese Emotionen äußern sich laut Thomas von Aquin in \"{}naturgegebenen\"{} Emotionen (wie Furcht - vor physischen Verletzungen und Vernichtung) und nicht naturgegebenen Emotionen (Furcht vor drohendem Übel, die in uns Kummer, Sorgen und Bedrückung auslösen). Naturgegebene Furcht entsteht also aufgrund des biologischen Selbst- und Arterhaltungstriebes. Die nicht-naturgegebene Furcht bezieht sich auf Bedrohung aller Güter, die der Mensch als intellektuelles Wesen hervorbringt, anstrebt oder verlieren kann \cite{forschner_thomas_2006}. Diese Emotionen hängen maßgeblich von Urteilen ab, die über die Vernunft kontrollierbar sind. Die Vernunft beurteilt demnach was tatsächlich passieren kann oder was als wertvoll erscheint und steuert damit den Wille. Dabei geht Thomas von Aquin nicht davon aus, dass die Vernunft unbegrenzt das sinnliche Fühlen und Streben aushebeln oder steuern kann, sondern die natürliche Furcht vor physischen Verletzungen und dem Tod immer vorhanden ist und gelegentlich bei sehr großen Unglücken auch ausbricht \cite{forschner_thomas_2006}.\\
Eine solche Betrachtung der Emotionen war sinnbildlich für die Zeit des Mittelalters in denen sich der christlich Glaube stark verbreitete. Generell wurden Emotionen, wie Gier, Völlerei, Lust, Ärger, Neid und Stolz als Sünden gesehen, welche durch die Vernunft gezügelt und vermieden werden sollten.  Emotionen wie Liebe, Hoffnung und Glauben wurden eher nicht als Emotionen interpretiert sondern hatten einen eher höheren Status, welcher mit Vernunft assoziiert wurde \cite{robert_c._solomon_philosophy_2008}.\\


Für Descartes (1596-1650) als Begründer der Modernen Philosophie stellten Emotionen aufgrund dessen, dass sie sich sowohl körperlich äußern (durch Erröten, Zittern und Erbleichen) als auch komplexe geistige Zustände sind \cite{perler_rene_2006}  eine Herausforderung für sein propagiertes Körper-Seele-Modell dar. Für ihn waren Körper und Geist grundsätzlich \"{}zwei  real  (nicht  nur  begrifflich)   verschiedene Substanzen [...], die je eigene Zustände haben. Nervenreizungen, Hirn-
zustände,  Schweißausbrüche  und  Bewegungen  der  Beine  sind  Modi,  
d.  h.  Zustandsformen,  der  körperlichen  Substanz. Mathematische  Gedanken, Wahrnehmungserlebnisse  und  viele  andere  geistige  Zustände hingegen sind Modi der denkenden Substanz. Da die beiden Substanzen real  verschieden  sind,  sind  auch  die  körperlichen  und  geistigen  Modi  real  verschieden\"{} \cite{perler_descartes:_2008}.
Um den Dualismus aufrecht zu erhalten ohne die Emotionen zu einem rein geistigen oder rein physiologischen Modi zu zu schreiben ordnete Descartes Emotionen dem Begriff für die Körper-Geist-Einheit und alle Zustände dieser Einheit zu.
Descartes betrachtet Emotionen demzufolge als einen psychophysischen Zustand des Geistes, welcher
\begin{itemize}
\item  mit  einer  bestimmten  Qualität  erlebt  wird,
\item einen äußeren Gegenstand in seiner Wirkung auf den Geist repräsentiert,
\item ihn  evaluiert  und
\item  ein  Repräsentationsmuster  festlegt,  das  auf  andere  
Gegenstände übertragbar ist.
\end{itemize}
Für ihn hatten Emotionen einen komplexen, \"{}propositionalen\"{} (aussagekräftigen) Inhalt. Durch das Empfinden von Angst zum Beispiel, nimmt er wahr, dass ein bestimmtes Ereignis gefährlich für ihn ist, daraus resultiert demzufolge dass das Ereignis als gefährlich wahrgenommen wird. Aufgrund dessen sieht er Emotionen als Funktionen, die Inhalte der Ereignisse bewerten \cite{amy_m._schmitter_17th_2016}. 
Anhand dieser Definition klassifiziert Descartes sechs grundlegende Emotionen: Verwunderung,  Liebe,  Hass,  Begehren,  Freude,  Traurigkeit. Eine Emotion wird laut ihm von einem äußeren Objekt hervorgerufen, welches auf die Sinnesorgane einwirkt und dadurch Nervenreizungen auslöst, welche wiederum Hirnzustände verursachen. Jeder einzelne Hirnzustand  korreliert  mit  einem  geistigen  Zustand,  d.  h.  mit  dem  geistigen Erleben einer Emotion. Die Unterscheidung zwischen den sechs grundlegenden Emotionen trifft Descartes aufgrund unterschiedlicher  Objekte 
(z.  B.  angenehme  oder  unangenehme,  häufig  oder  selten  vorhandene) die diese Emotionen hervorrufen, wie sie auf die  Sinnesorgane  einwirken und welche Hirnzustände sie dadurch hervorrufen. \cite{perler_descartes:_2008}\\


Im Gegensatz zu Descartes geht Baruch de Spinoza (1632-1677) nicht von einer restriktiven Teilung von Seele und Geist aus, vielmehr geht er von einer Einheit dieser zwei Teile aus. So erscheinen ihn Körper und Geist wie 2 Seiten einer Medaille. Innerhalb dieses Konstruktes definiert er Emotionen als \"{}verworrene Idee\"{} des Gemütszustand, die durch körperliche Reaktionen des Körpers beeinflusst wird und die Ideen und Handlungen des Geistes lenken kann \cite{moreau_imitation_2006}. 
Descartes Ausführungen folgend sind auch für Spinoza Emotionen natürliche Bewertungen, die nicht immer durch den Willen und der Vernunft gesteuert und unterdrückt werden können. Emotionen können zwar nicht immer unterdrückt werden, sie können allerdings durch Reflexion und der Ergründung der Ursache der Emotion in Zukunft beeinflusst werden. Noch dazu beschreibt Spinoza, dass Emotionen körperliche und kognitive Ausprägungen haben, diese aber nicht in jedem Fall \"{}gegeben\"{} sind sondern kulturell und erzieherisch geprägt und dadurch veränderbar sind. \cite{renz_spinoza:_2008} \\


Laut dem Aufklärer David Hume (1711–1776)  gehören Affekte zur Natur des Menschen, wobei diese meistens erst in sozialen Prozessen entstehen. Dabei teilt er die Emotionen in \"{}primäre\"{} bzw. \"{}ursprüngliche\"{}, in \"{}sekundäre\"{} bzw. \"{}reflexive Eindrücke\"{} sowie in heftige und ruhige und direkte und indirekte Affekte. Eindrücke sind für Hume intensive und lebhafte Vorstellungen. Hume unterscheidet primäre Affekte als Eindrücke der Sinneswahrnehmungen, welche im menschlichen Geist durch den Gebrauch der Sinne oder durch das Wahrnehmen des Körpers entstehen (z.B.  Hitze  oder  Kälte,  
Hunger  oder  Durst,  Lust  oder  Unlust), und sekundäre durch Eindrücke der Selbstwahrnehmung, welche aus den Sinneseindruck unmittelbar hervorgehen oder auf diesem Beruhen (z. B. Verlangen und Abneigung,  Hoffnung und Furcht). Sekundäre Affekte entstehen also aus der Reflexion der primären Affekte. Direkte Affekte sind laut Hume  „instinktive“ Affekt, die unmittelbar aus angenehmen bzw. unangenehmen Erlebnissen resultieren. Sie sind sozusagen das  "Vermögen  des  menschlichen  
Geistes,  das  Gute  zu  ergreifen  und  das  Übel  zu  meiden" (pp. 398) \cite{demmerling_hume:_2008}. Dazu zählt Hume die Affekte: Begehren,  Abscheu,  Schmerz,  Freude,  Hoffnung,  Furcht,  Verzweiflung und beruhigende Gewißheit. Indirekte Affekte beruhen auf komplexen Zusammenhängen zwischen Eindrücken und Vorstellungen. Dazu zählt Hume Stolz,  Kleinmut,  Ehrgeiz,  
Eitelkeit. Die Intensität spielt bei Hume ebenfalls eine Rolle so unterscheidet er ruhige Affekte wie dem Gefühl der Schönheit und Häßlichkeit oder heftige Affekte wie Liebe, Hass, Stolz und Niedergedrücktheit. Bemerkenswert ist dass Hume Vernunft von Affekt in folgendem Wortlaut voneinander trennt: 
"Was  wir  gewöhnlich  unter Affekt verstehen,  ist  eine  heftige  und  spürbare  Gefühlserregung im Geiste [...]. Unter Vernunft verstehen wir Gemütsbewegungen, die gleicher Art sind, wie die Affekte, die aber ruhiger wirken und keinen Aufruhr in der Gemütsverfassung hervorrufen. Diese Ruhe verleitet uns zu einem Irrtum über ihr Wesen, d. h. sie läßt uns dieselben als reine logische Leistungen unserer intellektuellen Vermögen erscheinen".
Wie \cite{demmerling_hume:_2008} richtig klarstellt bedeutet dies, dass alles was wir Fühlen und Denken  als Affekt oder eine Bewegung unseres Gemüts gesehen werden kann. \cite{demmerling_hume:_2008}\\


Für Kant (1724–1780)  waren Emotionen, wie auch im Stoizismus eher Störfaktoren, die verunftswidrige Handlungen hervorrufen. So schrieb er: "Affekten    und    Leidenschaften    
unterworfen zu sein, wohl immer Krankheit des Gemüts ist; weil beides die Herrschaft der Vernunft ausschließt".  Er unterscheidet dabei Affekte von Leidenschaften (pathos). So bezeichnen Affekte „das Gefühl einer Lust oder Unlust  im  gegenwärtigen  Zustande,  welches  im  Subject  die  Überlegung (die  Vernunftvorstellung,  ob  man  sich  ihm  überlassen  oder  weigern  solle)  nicht  aufkommen  läßt“  (ApH  251). Sie entstehen aus der Überraschung heraus, die das menschliche Gemüt aus der Fassung bringen. Leidenschaften waren für ihn langanhaltende von der Vernunft schwer oder garnicht bezwingbare Neigung.\\


Nietzsche (1844–1900) hingegen kritisierte, diese Vorstellung dass die einzig wahre (stoische) Vernunft unser Leben bestimmen sollte - vielmehr wertete er die Emotionen hinsichtlich ihrer Funktionen für das menschliche Handeln auf. So beschrieb er, dass "Denken, Schliessen und Rechnen" ohne die "regulierenden unbewußt-sicherführenden Triebe" dazu führten, dass Menschen die einfachsten Verrichtungen nur "ungelenk" mit "entsetzlicher Schwere" ausführen konnten \cite{nietzsche_zur_2012}. In seinen Theorien gibt es keine absolute Geistigkeit, keine absolute Vernunft, da es nur ein perspektivisches Sehen und Erkennen gibt, welches "den aktiven und interpretierenden Kräften" unterliegt. Er behauptet dabei, dass der größte Teil unseres geistigen Wirkens unbewußt verläuft und dabei verschiedene Triebe "mit einander kämpfen" um sich "fühlbar zu machen"\cite{nietzsche_frohliche_1887}.\cite{werner_stegmaier_nietzsche:_nodate}\\
Da Anfang des 19. Jahrhunderts sich aus der Philosophie die Psychologie abspaltete wurden Emotionen die eher als Forschungsgegenstand der Psychologie gesehen, weil diese ja maßgeblich für das menschliche Leben und Handeln maßgeblich zu sein schienen. Theorien über Emotionen sind in diesem Jahrhundert also rar gesät.\\ 


Alfred North Whitehead (1861–1947) sah Emotionen immer in Verbindung mit Erfahrungen. Im Gegensatz zu der damaligen verbreiteten Meinung, dass der Prozess der Erfahrungen mit klaren, einfachen und unbewerteten Sinnesdaten beginnen, dann mit einer Vorstellung von komplexen Gegenständen verbunden, beurteilt und bewertet werden und danach die subjektive Emotion ausgelöst wird, war Withehead der Meinung, dass Emotionen während des gesamten Prozesses eine maßgebliche Rolle spielt. Für ihn beginnt ein solcher Prozess der Wahrnehmung mit einer zunächst undeutlichen, komplexen und unkontrollierten Emotion und wird dann durch komplexe innere Prozesse und Erfahrungsanalysen  und -transformationen "bewußt" interpretiert. Das erfahrende Subjekt unterliegt also Prozessen in denen viele Emotionen zusammenkommen und eine "einzigartige momentane Erfahrungseinheit bilden"\cite{maria-sibylla_lotter_whitehead:_nodate}. Diese Erfahrungen wirken sich dabei nicht nur auf die momentane Situation sondern auch auf zukünftige Situationen aus. Im original text bescheibt er alle Erfahrungen \cite{alfred_north_whitehead_adventures_1967}: „The  basis  of  experience  is  emotional.  Stated  more  generally,  the  basic  fact  is  the  rise  of  an  affective  tone  originating  from  things whose relevance is given." Das heißt jedes gegebene "Datum" ist objektiv und bestimmt in sich selbst. Aber das Subjekt nimmt es auf eine bestimmte emotionale Art wahr, die sich innerhalb gewisser Grenzen gegeben von der Realität bewegt. Dies bedeutet auch, dass jede subjektive Einheit sich von der anderen unterscheidet und dass jedes Subjekt die objektive Realität ("objective datum") anders wahrnimmt bzw. "fühlt" \cite{shaviro_without_2012}.\\   


Nach Brentano sind Gefühle psychologische oder geistige Episoden, die sowohl eine affektive als auch eine intellektuelle wahrnehmungs-, erinnerungs- oder erwartungsmäßige Komponente haben. Wobei die affektive Seite auf die intellektuelle Seite aufgefasst wird. \cite{moser_philosophie_2016} Brentano unterteilte 3 Klassen Gefühl, Wille und Urteil. Wobei er Wille dem Streben von Aristoteles zuordnet und Gefühle als Liebe, Hass, Lust und Unlust beschreibt. Der Wille und die Gefühle bilden dabei eine Einheit, denn es besteht keine "scharfe Grenze" zwischen Gefühl und dem Wille. So führt er aus, dass "Ehe Jemand die Erkenntniss oder wenigstens die Vermutung gewonnen hat das gewisse Phänomene der Liebe und des Verlangens die geliebten Gegenstände unmittelbar oder mittelbar als Folge nach sich ziehen, ist ein Wollen für ihn unmöglich"\cite{franz_clemens_brentano_psychologie_1874}.\\ 


Wittgenstein unterscheidet Emotionen und Empfindungen. Emotionen haben einen zeitlichen Verlauf aber keinen Ort. Man kann sagen wie lang und wann man traurig war allerdings keine Körperstelle für die Trauer benennen. Schmerzempfindungen haben hingegen eine Lokalisierung und darüber hinaus liefern sie Informationen über die Außenwelt (Geschmackserlebnis, Tastempfinden), diese haben also einen Kausalen Zusammenhang zu einem Objekt in der Außenwelt. Man kann anhand der Empfindung auf die Ursache (schmackhafte Mahlzeit, Kugel) schließen. Im Gegensatz dazu richtet sich eine Emotion auf ein Objekt, dieses muss aber nicht gleich die Ursache sein. Ein Objekt kann Angst auslösen obwohl es nicht gefährlich in dem Sinne ist. Emotionen beeinflussen dabei das Denken und werden in einer für sie charakteristischen Weise ausgedrückt. Anhand der Beschreibung von Emotionen in unserem Sprachgebrauch unterscheidet er zwischen Emotionen in der 3. Person und in der 1. Person. Er sieht dabei die emotionale Körperreaktion als Voraussetzung damit ein differentiertes sprachliches Emotionsverhalten überhaupt entstehen kann. So handelt sich bei Emotionen in der 3. Person um Mitteilungen einer Beschreibung des emotionalen Verhaltens anderer Menschen. In der ersten Person hingegen um den Ausdruck der Emotion selbst. Es ist also keine Beschreibung des eigenen Zustandes sondern einfach nur ein "unvermittelter" Ausruf. Damit steht er konträr zu den 3 maßgeblichen Emotionstheorien.
Der mentalistischen, weil es sich bei Emotionsausdrücken in der ersten Person um unmittelbare, reaktive Handlungen, denen eine "Gewissheit" im Körper zugrunde liegt. Für ihn sind Emotionen Ausdrücke abhängig vom menschlichen Verhalten und dem Kontext in dem sich der Mensch befindet und nicht die Beschreibung eines inneren Zustandes. Auch bei der Beschreibung des emotionalen Ausdrucks anderen ist die zuschreibung einer Emotion keine innere Zustandsbeschreibung sondern diese Beschreibung entsteht aus der Annahme dass der Gegenüber ein Seele hat und dem Verstehen, dass der unmittelbar wahrgenommene emotionale Zustand in einem für den anderen menschlichen Verhalten und Kontext steht. Im Gegensatz zur behaviouristeschen Emotionstheorie sind für ihn Emotionen keine Beschreibungen, man "fürchtet sich" resultiert nicht aus einer Beobachtung meines verhaltens sondern es ist eine Emotion die ich habe. Da es hier keine kausale Beziehung zwischen Reiz und emotionaler Reaktion gibt, sondern die emotionalen Verhalten bestimmte "Muster" sind die sich aus den verschiedensten Kontexten herausbilden. Auch bei der Beobachtung von Emotionen bei anderen wird für ihn nicht das Verhalten der Person analysiert sondern die Gefühle des Gegenübers anhand von äußeren Kriterien erfasst und innerhalb eines Kontextes analysiert. Diese Kriterien stellen aber keine notwendigen Bedingungen dar. So kann Lachen nicht immer ein  Zeichen von Freude sein sondern auch Unsicherheit oder Zynismus signalisieren. Sie sind auch kein Indiz oder Symptom in dem Sinne, da jeder seine Emotionen wie Freude, Trauer und Zorn aus der Individualität unter einem kulturellen Hintergrund heraus empfindet. Die physiologische Emotionstheorie verneint er weil Emotionen zwar mit einem bestimmten Körperausdruck einhergehen doch sind sie keine notwendigen Bedingung. Er defniert dabei die "primitive Reaktion", welche eine Geste, ein bestimmter Gesichtsausdruck oder ein Wort sein kann welche die eigene situatuon unmittelbar ohne Interpretation und Erkenntnis zum Ausdruck bringt. Diese instinktiven bzw. primitiven Reaktionen sind dann die Vorraussetzung für das entstehen von Sprache über Emotionen. Erst an dieser Stelle kommt die Vernunft die versucht das eigene Handeln zu begrundne und bezweifeln, welche dann zu irreführenden Annahmen führen kann. Emotionen sind für ihn "Muster" die in einen bestimmten Kontext auftreten. Der Ausdruck einer solchen Emotion kann variable sein und sich in unterschiedliche Muster anderer passender emotionaler Ausdrücke einfügen. \cite{gunter_gebauer_wittgenstein:_2008}\\


Martin Heidegger (1889–1976) unterteilt die Befindlichkeit der Menschen in Stimmung und Emotionen. Wobei Stimmungen passiv und aufgrund der "Last" des weltlichen Seins entsteht und nicht objekt-bezogen sind. Emotionen sind für ihn Ausdrücke, die aus der Beziehung zwischen der Situation und einem selbst entsteht. So wird eine Bedrohlichkeit durch die Emotion der Furcht zugänglich. Manchen Emotionen gehen bestimmte Stimmungen voraus, beziehungsweise können die bestimmte Emotionen unterdrücken oder verstärken. Zum Beispiel könnte eine düstere Stimmung freudige Emotionen und die durch sie erschließbaren positiven Eigenschaften für das innerweltliche "Seiende" unmöglich empfunden werden. Er betrachtet Emotionen dabei als Eigenständige Phänomene die nicht auf andere existenzielle Phänomene wie Wünsche, Überzeugungen, Werturteile, Lust oder Unlust oder eine Kombination aus diesen reduziert werden können. Er analysiert Emotionen anhand dreier Aspekte dem Worin, Wovor und dem Worum. In dem Worin wird analysiert welches Objekt diese Emotion hervorruft. Er geht dabei von einem gerichteten Zusammenhang zwischen innerem Empfinden und äußerer Welt aus. Der Zustand bzw. die Emotion (das fürchten) in welchen man sich befindet beantwortet die Frage des Worin und das Worum wird erzeugt aus der Interpretation wovor man sich fürchtet, allerdings immer in der Beziehung zu dem (individuellen) selbst. \cite{barbara_merker_heidegger_2008}\\


Für Jean-Paul Sartre (1905–1980) sind Emotionen Bewußtseinszustände oder Bewußtseinsphänomen, welche eine funktionale Rolle im menschlichen Handeln haben. Er betrachtet dabei Emotionen nich als Wechselspiel zischen Unbewußten und Bewußten , sondern als intrinsischen Zustand des Bewußtseins als ganzes. Er mißt Emotionen eine ihnen eigene Bedeutung besitzen und die Eigenschaft zur eine Modifikation der Weltwahrnehmung bzw. ein Einfärben der Welt, wobei Emotionen den Dingen in der Welt eine Bedeutung verleihen. Sartre besteht darauf, dass Emotionen verbunden mit ihrem Objekt sind und ohne dieses nicht existieren würden. Die Emotion fügt deshalb auch der Welt nichts Neues 
hinzu.  Sie  kann  nicht  als  die  Einfärbung  einer  ansonsten  neutralen  Welt  
oder  eines  gleichsam  kalten  Objektes  verstanden  werden.
So sagt er "  Die  Emotion  ist  eine  bestimmte  Weise,  die  Welt  zu  verstehen".\cite{jean-pierre_wils_sartre:_2008} \\

 Susanne K. Langer (1895–1985) (basierend auf Whithead) widmet sich in ihren philosophischen Ausführungen dem Fühlen. Für sie entsteht das Fühlen in der spezifischen Existienzform lebendiger Prozessualisierung.  Ein Hauptstrang der Evolution des Fühlens besteht darin, dass es eine Funktion für die Steuerung des Verhaltens gewinnt. Der  Begriff  des  „Fühlens“  als  Grundbegriff  ihrer  Philosophie  des  Geistes wird so definiert, dass durch ihn der gesamte Umfang dessen, was von uns gefühlt werden kann - angefangen von einzelnen Empfindungen, 
Schmerz-  und  Lustgefühlen,  dem  Spüren  von  Vitalität  und  Stimmungen  
bis hin zum Gefühl eigener Identität, komplexen Emotionen, intellektuel-
len  Spannungen  und  unserem  bewussten  Denken  -  bezeichnet  wird. 
 Dabei bezeichnet sie bestimmte wiederkehrende Verhaltensabläufe als Einheiten bzw. Akte. Diese sind raum-zeitliche und energetische Naturabläufe / Prozesse. Sie werden durch eine Ausgangsspannung iniitiert und verbrauchen diese Spannung, Sie haben verschiedenen Phasen, gehen von einem Impulse aus, gehen in die Beschleunigungsphase über, erreichen einen Höhepunkt und münden in eine Phase des Verbrauchens, welches mit den Abbruch der Aktivität endet. Diese Aktivitäten oder Prozesse bestehen innerhalb eines sich selbst aufrechterhaltenden geschehen, welches das Leben charakterisiert. "Ein  Lebewesen  ist  eine  riesige  Verkettung  und  Verflechtung  von  Akten,  die  sich  reproduzieren,  wechselseitig  stützen,  beeinflussen,  zu  größeren  Einheiten  integrieren  oder  auch  blockieren.  Das  komplexe  Zusammenspiel  der  wechselseitig  aufeinander  bezogenen Akte, bestimmt den Organismus in allen Hinsichten. " Dabei ist die Entstehung von solchen Akten ein natürlicher kausaler Zusammenhang, allerdings ist dieser Zusammenhang so komplex dass er sich dem Verständnis entzieht. Das Fühlen entsteht bei besonders intensiven Aktivitäten, in denen Bewusstsein als ein neuer Aspekt auftaucht. Dabei ist dieses Bewusstsein evolutionär entstanden.  Das  Fühlen ist  eine  Qualität,  die  in  bestimmten  Akten  auftaucht und  mit  ihnen  wieder  verschwindet.  Langer  spricht  davon,  dass  in  bestimmten Akten eine „psychische Phase“ (M I, 21) entsteht. Hinsichtlich  der  physiologischen  Grundlagen  formuliert  Langer  die Hypothese,  dass  das  Entstehen  einer  psychischen  Phase  wesentlich  mit  
dem  energetischen  Charakter  der  Akte  zusammenhängt.  Demnach  „brechen“  neurophysiologische  Akte  ab  einer  bestimmten  Intensitätsschwelle  
in die psychische Phase „durch“ und werden gefühlt. Das Auftauchen der 
psychischen  Phase  ist  allerdings  kein  punktuelles  Ereignis,  sondern  ein  
allmähliches  Anschwellen,  für  das  man  keinen  isolierten  Grenzwert  bestimmen  kann.  Langer  spricht  von  einer  „fluktuierenden  Empfindungs-
grenze“. (M I, 22) 

Auch Giovanna Colombetti (2014) beschreibt, dass Emotionen nicht nur kognitive Komponenten sondern auch körperliche Komponenten haben. Die spricht von der enaktiven Ansatz Emotionen zu beschreiben. Im Gegensatz zum sehr weit verbreitet Kognitivismus bei der der Körper und dessen Reaktionen von der Kognition gesteuert und beeinflusst werden, vertritt der Enaktivismus die Idee, dass das lebendige Wesen bestehend aus Körper, Geist und Welt zur Kognition befähigt. Das hat zur Folge dass Kognition nicht etwas ist was passiert sondern viel mehr was wir tun. In diesem Zusammenhang spricht Colombetti das das Verhalten von lebenden selbst-organisierten Systemen aus rekurrenten, wechselseitigen Einflüssen aus der Interaktion zwischen Körper, Geist und Umwelt entstehen und nicht durch gerichtete bewußte Entscheidungen die nur im Gehirn getroffen werden.Der Körper spielt im Enaktivism auch innerhalb der menschlichen Erfahrungen eine wichtige Rolle. Denn das Verstehen des Geistes erfordert eine Untersuchung und Ergründung unserer verkörperten Natur auf dem physischen und erfahrungsgemäßen Level. 

In diesem Zusammenhang sind für Comlombetti Emotion Episoden, die selbst-organisierende Muster des Organismen bereitstellen. Wie Kognition entstehen Emotionen aus rekurrenten, wechselseitigen Prozessen (neural, muscular, usw) innerhalb von lebenden Systemen. Die beschreibt selbst-organisierende emotionale Episoden als sehr variable, weil die Prozesse, die diese Episode auslösen können, abhängig vom Kontext unterschiedlich organisiert und aktiviert werden können. Damit ist eine Emotion und deren Vielfalt abhängig vom Zustand des Organismus' und seinem evolutionären und historischen Entwicklungen abhängig.

Das bedeutet Emotionen sind frei von internen Kausalitäten, wie Affektprogrammen oder Sequenzen von kognitiven Bewertungen. Außerdem würde es keine Basisemotionen geben, die Basiseinheiten bilden und dann die Zusammensetzen von "höheren" oder komplexeren Emotionen ermöglichen. Vielmehr sind alle emotionalen Episoden komplexe, flexible und variable selbst-organisierte Muster, wobei einige dieser Muster innerhalb verschiedener Kulturen gleich sind und andere Muster nur unter bestimmten Kontexten oder spezifischen Individuen entstehen. Im Unterschied zu vielen Theorien beschreibt Comlombetti, dass selbst einfach Organismen Emotionen haben können, wobei sie damit vor allem Theorien aus Neurokognition und Neurobiologie stützt und kognitivistische und sprachbasierte Ansätze zur Beschreibung von Emotionen verneint.


% \newpage
% \begin{landscape}

% \begin{center}
%   \begin{tabular}{ | p{2cm} | p{2cm} | p{2cm} | p{2cm} |p{2cm} | p{2cm} |}
%     \hline
%     Theorie & Wer hat Emotionen? & Was passiert während einer Emotion? &Wann treten Emotionen auf? & Wie werden Sie ausgelöst? & Warum wurden sie ausgelöst? \\ \hline
%     Aristoteles & Menschen & Der Mensch wird von der Emotion kontrolliert. & Können immer auftreten.& Wenn jemand etwas getan hat was einen selbst oder Freunde betrifft. & Weil etwas unerwartet schlecht oder gut gelaufen ist.\\
%     &&Urteile ändern sich.&&&\\
%     &&Schmerz oder Vergnügen wird empfunden. &&&\\ \hline
%     7 & 8 & 9 \\
%     \hline
%   \end{tabular}
% \end{center}

% \end{landscape}






