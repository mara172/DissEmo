\section{Biologisch}
Die Emotion als kognitiver Prozess ist ein wichtiger Bestandteil unseres Lebens. Es gibt
ein facettenreiches Spektrum an Emotionen zum Beispiel Angst, Neid, Freude, Wohl-
befinden, Ärger, Wut und viele andere, welche das Handeln und Verhalten der Menschen
beeinflussen. So ist das Ziel eines Menschen ein glückliches, erfülltes Leben zuführen und
Gefahren für Leib und Seele aus dem Weg zu gehen. Doch „obwohl es vielerlei Emotionen
gibt und diese mit zahlreichen körperlichen Prozessen einhergehen, existiert bisher keine
exakte wissenschaftliche Definition des Begriff Emotion“ (Kandel et al., 1996).
Welchen Einfluss Emotionen auf den Menschen und dessen Verhalten haben wurde in
vielen Studien untersucht. Peters et al. (2006) differenzierte in seiner Studie die ver-
schiedenen Rollen, die Emotionen spielen, wenn Menschen Entscheidungen treffen. Er
publizierte, dass Emotionen zu Informationen führen und einen selektiven Fokus für die
Aufmerksamkeit setzen können. Zudem kann eine Emotion als Motivator fungieren und
eine Bewertung alternativer Verhaltensmöglichkeiten treffen. Dies zeigt, dass die Emotion
innerhalb der Kognition und des Verhaltens eine wichtige und entscheidende Rolle, die

%Abb. 2.1: Darstellung der Zwei-Faktoren-Theorie von Schachter und Singer (1962).

es unerläßlich macht, sich mit den Prinzipien und der Wirkungsweise der emotionalen
Verarbeitung auseinanderzusetzen.
Für die Erforschung von Emotionen gibt es einen Ausgangspunkt, welcher in den Gefüh-
len eines Menschen, die dieser subjektiv und bewusst wahrnimmt, liegt. Die Erforschung
dieser Gefühle wurde im letzten Jahrhundert verstärkt. Demzufolge und durch die Weiter-
entwicklung der Technik erhielt die Interpretation von Emotionen und deren kognitiven
Bedeutung einen andere Sichtweise.
Eine der einflussreichsten Theorien der Emotionsforschung ist die „Zwei-Faktoren-
Theorie“ der Emotionen von Schachter und Singer (1962). Sie definiert drei Faktoren, von
denen eine Emotion beeinflusst wird. Der erste Faktor ist die emotionsauslösende Situati-
on, der zweite Faktor ist die unspezifische, physiologische Aktivierung von Neuronen im
Gehirn und der dritte Faktor ist die Betrachtung der emotionale Situation als Ursache für
die physiologische Aktivierung. Dabei soll die Höhe der physiologischen Aktivierung die
Intensität und Interpretation der Qualität von Emotionen beeinflussen (siehe Abb. 2.1).
(Müsseler und Prinz, 2002)
Mit verbesserten Techniken in der Gehirnforschung und neuen Erkenntnissen in der Neu-
rophysiologie entwickelte LeDoux (2001)auf Grundlage diverser Theorien ein Modell,
welches sowohl kognitive als auch biologische Erkenntnisse zur Entstehung von Emo-
tionen integriert. LeDoux untersuchte die Furchtreaktionen sowie deren Ursprung und
fand heraus, dass die Amygdala, eine kleine Region im Vorderhirn, zentraler Bestandteil
des neuralen System (siehe Abb. 2.2) ist. Die Bedeutung der Amygdala für die Angst-
konditionierung und den damit verbundenen Angstreaktionen wurde in diesen Studien
bewiesen.

%Abb. 2.2: Die Amygdala ist ein zentraler Bestandteil in der Emotionsverarbeitung
%(Brain stories, 2011).

Spätere Forschungen zeigen, dass die Amygdala nicht nur der zentrale Bestandteil im
System der Angstverarbeitung ist, sondern auch andere emotionale Verarbeitungssysteme
diese als zentrale Verarbeitungsstation besitzen (Everitt et al., 2000). Ein solches Ver-
arbeitungssystem wird auch Basisnetzwerk der Emotionen bezeichnet. Panksepp (2010)
definiert in einer Studie über Depressionen sechs Kriterien für diese Basisnetzwerke:
\begin{itemize}
\item Sie generieren charakteristische instinktive Verhaltensmuster.
\item Sie werden durch eine begrenztes Set von unkonditionierten Stimuli aktiviert.
\item Die resultierenden Erregungen überdauern herbeigeführte Umstände.
\item Emotionale Erregungen verknüpfen beziehungsweise regulieren verschiedene sen-
sorische Eingaben im Gehirn.
\item Sie kontrollieren das Lernen und helfen höhere Gehirnfunktionen zu realisieren.
\item Höhere Gehirnareale können emotionale Erregungen regulieren.
\end{itemize}

Emotionen dieser Verarbeitungssysteme unterscheiden sich sowohl in der Intensität,
Komplexität als auch in den beteiligten Gehirnarealen. Deshalb werden Emotionen an-
hand ihrer Eigenschaften verschiedenen Begriffen zugeordnet. So werden kurze und star-
ke Emotionszustände, die starke Verhaltenstendenzen besitzen Affekte genannt. Als Emo-
tion werden bewertende Stellungnahmen zu Umweltereignissen, welche der Koordination
von verschiedenen physischen und psychischen Teilsystemen bedürfen, bezeichnet. Im
Gegensatz zu Emotionen werden emotionale Zustände von geringerer Intensivität, länge-
rer Dauer mit einem Fehlen der Objektbezogenheit als Stimmungen definiert. Ein Gefühl

%Abb. 2.3: In der sogenannten Skinnerbox werden Experimente der Angstkonditionierung
%durchgeführt (Ricker, 2011).

wiederum bezeichnet die erlebnisbezogene Seite einer Emotion, zum Beispiel Wut oder
Angst. (Müsseler und Prinz, 2002)
Doch Emotionen können nicht nur anhand ihrer Eigenschaften unterschiedlich katego-
risiert werden sondern auch anhand ihrer prozeduralen Verarbeitung. Panksepp (2010)
unterteilt in diesem Zusammenhang die emotionalen Prozesse in drei Klassen. Affekte
werden im primären Prozess verarbeitet. Sie bezeichnen Urinstinkte, wie Fluchtreflexe
und Angstreaktionen, die genetisch vererbt werden. Affekte sind unkonditionierte Re-
aktionen des Emotionssystems. Sekundäre Prozesse sind erlernte einfache Emotionen,
die das Individuum durch klassisches oder operantes Konditionieren (siehe auch Kapitel
2.5) erwerben kann (Müsseler und Prinz, 2002). Emotionen, die ein vielschichtigeres
Denken und Planen erfordern werden dem tertiären Prozess zugeordnet. Diese Emotio-
nen werden reflektiert und reguliert, außerdem tritt diese Art von Emotionen nur beim
Menschen auf. Die Grundlagen des menschlichen Lebens bilden die primären Prozesse.
Ein neurochemisches Ungleichgewicht in der emotionalen Verarbeitung dieser Prozesse
kann zu einer psychiatrischen Auffälligkeit führen, da die dynamischen sekundären und
tertiären Prozesse mit dem System der primären Prozesse verbunden sind. Es wird also
davon ausgegangen, dass diverse höhere psychologische Funktionen auf der Verarbeitung
einfacher emotionaler Reaktionen beruhen. (Panksepp, 2010)

%Abb. 2.4: Die Darstellung veranschaulicht die zwei Verarbeitungspfade zur Amygdala
%(LeDoux, 2001).

% 2.5 Konditionierung
% Konditionierung bezeichnet einen Lernprozess, welcher bei Menschen und Tieren beob-
% achtet werden kann. Dabei bezeichnet „ Lernen [...] ein(en) Prozess, der als Ergebnis von
% Erfahrungen relativ langfristige Änderungen im Verhaltenpotenzial erzeugt“ (Definition
% Müsseler und Prinz (2002)).
% Es gibt das klassische Konditionieren, welches unkonditionierte Stimuli (US) mit neutra-
% len, konditionierte Stimuli (CS) paart. Voraussetzung eines solchen Lernprozess ist, dass
% der US (z.B. Futter) und der CS (z.B. ein Lichtsignal) mehrmals gemeinsam repräsentiert
% wird. (Müsseler und Prinz, 2002)
% Im Gegensatz zum klassischen Konditionieren beinhaltet das operante Konditionieren
% nicht nur einen Reaktion auf einen Reiz, sondern ein (instrumentelles) Verhalten, dass
% ein Ereignis in der Umwelt herbeiführt (z.B. das Betätigen eines Hebels zum Öffnen
% einer Futterdose). Diese Ereignis würde ohne das Verhalten des Versuchsobjektes nicht
% eintreten. (Müsseler und Prinz, 2002)
\subsection{ Amygdala}
In der Angstkonditionierung wird ein Tier, zum Beispiel eine Ratte, innerhalb eines Ver-
suchsfeldes platziert (siehe Abbildung 2.3). Anschließend wird zuerst ein Ton repräsentiert 


%Abb. 2.5: Forschungen zur Funktion des Hypothalamus.
und kurze Zeit später ein elektrischer Schlag ausgelöst, der eine Angstreaktion
hervorruft. Nach mehreren Wiederholungen, löst allein der Ton eine Furchtreaktion bei
der Ratte aus. Diese und ähnliche Experimenten der Angstkonditionierung zeigten, dass
Tiere, die eine Beschädigung der Amygdala besaßen, keine konditionierten Reaktionen
auf die Präsentation des Tons entwickelten. Es wurde somit bewiesen, dass die Amygdala
als Steuereinheit der Angstreaktionen fungiert. (LeDoux, 2001)
Die entwickelten Furchtreaktionen können je nach Situation und Ausgangszustand so-
wohl bewusst als auch unbewusst stattfinden. Je nachdem, von welchem der beiden Ver-
arbeitungspfade zur Amygdala diese generiert werden. LeDoux nannte diese zwei Wege,
zum Einen den „hohen“ Pfad, welcher vom sensorischen Thalamus über den sensorischen
Kortex zur Amygdala führt und zum Anderen den „niederen“ Pfad, welcher Eingaben
vom sensorischen Thalamus direkt an die Amygdala projiziert (siehe Abb. 2.4). Ein
emotionalerer Reiz wird demzufolge zunächst vom sensorischen Thalamus verarbeitet.
Bestandteile des sensorischen Thalamus sind zum Beispiel der Visuelle oder Rhinale
Kortex, welche visuelle beziehungsweise geschmackliche Eingaben verarbeiten. Diese
Informationen können anschließend auf dem kürzeren und direkten Weg an die Amygdala
weitergegeben werden, wobei die Repräsentation des Reizes innerhalb dieses Verarbei-
tungspfades eher rudimentär und grob ist. Desweiteren kann die Informationsweiterlei-
tung an die indirekte Bahn über den sensorischen Kortex erfolgen. Dieser Pfad liefert
eine genaue aber auch langsamere Repräsentation des Reizes. Durch die Verarbeitung in
höheren Gehirnarealen werden Objekte erkannt, eingeordnet und die aktuelle Situation
analysiert. Am Beispiel einer Gefahrensituation ermöglicht die direkte Bahn ein beson-
ders schnelles, reflexartiges Handeln (wie ansteigende Herzfrequenz, Adrenalinausstoss,

%Abb. 2.6: Im Zwischenhirn befindet sich der Hypothalamus, der eine sehr starke Verbin-
%dung zur Amygdala besitzt (Focalaxis, 2011).

Weglaufen) und die indirekte Bahn die Bewertung und Analyse der Situation (Erkennen
was ist die Gefahr, wie gefährlich ist die Situation). (LeDoux, 2001)
\subsection{ Hypothalamus}
Die Amygdala spielt neben der Angstkonditionierung auch eine wichtige Rolle beim Ler-
nen durch positive und appetitive (d.h. belohnenden) Ereignisse. Dies bewiesen Nakamura
et al. (1987) in Experimenten, die eine Verabreichung von verschiedenen Nahrungsmitteln
als Belohnung verwendeten. In diesen Studien zeigten sie, dass besonders die Verbin-
dung von Amygdala und Hypothalamus für das Gefühl von Hunger und Sättigung einer
Nahrung von Bedeutung ist. Während des Experimentes wurden Neuronen des Lateralen
Hypothalamus von Ratten abgeleitet und deren Aktivität aufgezeichnet(siehe Abb. 2.5a).
Es wurden verschiedene Gerüche, wie Orange oder Traube als positiven CS eingesetzt.
Zudem fundierte ein Schlag oder Kneifen in den Schwanz als negativer CS. Durch Prä-
sentation der entsprechenden CS erfolgte das Konditionieren der Ratte, sodass diese beim
Auftreten des entsprechenden Reizes die verabreichte Glukoselösung zu sich nahm.
Der Lernprozess mit einem appetitiven Stimulus kann vereinfacht, wie in Abb. 2.5b,
dargestellt werden. Ein beliebiger Stimulus wird mit einer primären Belohnung (Futter

%Abb. 2.7: Experimenteller Aufbau zur Dopaminausschüttung. Links der Versuchsaufbau
%und rechts die Aktivität der Dopaminzelle (Schultz et al., 1998).

oder Wasser) durch ständige Wiederholung konditioniert. Der konditionierte Stimulus
löst durch die Erwartung einer Belohnung einen internen angeregten Zustand aus. Meist
korrespondiert diese Erwartung mit Hunger oder Durst. Als Folge der Erwartungshaltung
erfolgt anschließend die Verhaltensreaktion, das heißt Enttäuschung bei Ausbleiben und
Zufriedenheit bei Belohnungsvergabe. Nach Nakamura et al., werden diese internen Zu-
stände durch den lateralen Hypothalamus gesteuert, denn Neuronen dieses Gehirnareals
reagierten selektiv auf die verschiedenen konditionierten Stimuli.
Die Reaktionen der Ratte, zum Beispiel Hunger, Durst oder Blutdruckänderungen, er-
klären sich aus der Funktion des Hypothalamus. Dieser befindet sich im Zwischenhirn,
dargestellt in Abbildung 2.6 und steuert die vegetativen Funktionen des Gehirns, un-
ter anderem die Aufrechterhaltung von Temperatur, Blutdruck und Osmolarität, die zur
Homöostasie zusammengefasst werden, oder die Regulation der Nahrungs- und Wasser-
aufnahme. Zusätzlich veranlasst der Hypothalamus die Bildung von sogenannte Effekt-
hormonen, wie Dopamin, welche bei Lernaufgaben und der Konditionierung ein zentrale
Aufgabe besitzt.
\subsection{ Basalganglien}
Der Dopaminhaushalt, der sich während eines Lernprozesses ändert, wird durch die
Basalganglien reguliert (Schultz et al., 1998). Um die Ausschüttung und Senkung von
Effekthormonen insbesondere von Dopaminen zu erklären, führten Schultz et al. in ihrer
Studie eine Verhaltensaufgabe durch (siehe Abb.2.7). In diesem Experiment löste das Tier
16

%Abb. 2.8: Darstellung der Basalganglien und ihren dazugehörigen Gehirnarealen (The
%Dana Fondation, 2011).
in einem selbst gewählten Moment einen berührungsempfindlichen Schalter aus und griff
in die Futterbox, um daraus ein Stückchen verstecktes Futter einzusammeln. Die Erregung
der Dopaminzelle, nachdem das Tier in die Futterbox gegriffen hat, ist in Abbildung 2.7
dargestellt. Der Movement-onset bezeichnet den Moment indem das Tier den Schalter
bewegt hat.
In diesem Zusammenhang fanden Schultz et al. heraus, dass die Basalganglien Dopamine
nicht nur regulieren sondern auch einen Fehler in der Vorhersage von Belohnungen ko-
dieren. Eine Belohnungsvorhersage erfolgt durch die Aktivität der Dopaminneurone wäh-
rend der Präsentation des CS. Erfolgte nach dieser Präsentation eine Belohnung, initierten
die Neuronen der Basalganglien einen Dopaminausstoß. Wurde jedoch die Belohnung
ausgelassen, reagierten die Zellen mit einer Senkung des Dopamins. Diese Senkung stellt
die Detektion des Fehlers in der Belohnungsvorhersage dar.
2.5.4 Orbitofrontaler Kortex
Neuronen in den Basalganglien können zwar Belohnungsvorhersagen treffen, jedoch kön-
nen sie nicht zwischen den verschiedenen Belohnungen (aversiv oder appetitiv) unter-
scheiden. Aus diesem Grund untersuchten Schultz et al., in welchem Gehirnareal eine
solche Verarbeitung stattfindet. Hier wurden sie im Orbitofrontalen Kortex fündig. Bei
Tests mit unterschiedlichen Belohnungen zeigten 70% der gemessenen Neuronen eine
korrespondierende Antwort. Sie antworteten entweder bevorzugt auf die eine oder auf
die andere Belohnung. Daraus lässt sich schließen, dass diese Neuronen Details von Be-
lohnungen kodieren. Verbindungen mit dem sensorischen Thalamus und der Amygdala
ermöglichen dieser Hirnregion eine subjektive Wertebildung für Objekte. Die Übertra-
gung einer solchen Wertigkeit erfolgt unter anderen an die Basalganglien, in welchen
anschließend entschlüsselt wird ob ein positiver oder negativer Reiz vorliegt und die Do-
paminausschüttung oder Dopaminsenkung unterdrückt oder veranlasst werden kann.