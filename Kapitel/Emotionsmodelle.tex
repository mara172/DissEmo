\chapter{Exisitierende Emotionsmodelle}

Für die Simulation menschlicher Emotionen existieren in der Literatur viele Emotionsmodelle, die das
emotionale Verhalten von Menschen erklären bzw. simulieren soll (Marsella and Gratch, 2010). Die
Modelle sind unterschiedlich komplex und verfolgen unterschiedliche theoretische Ansätze der
Psychologie (Marsella and Gratch, 2014). Allerdings sind wenige dieser Modelle mit realen Messdaten
aus psychologischen Experimenten validiert worden (Marinier et al., 2013). Das in der Dissertation
entwickelte Emotionsmodell soll sowohl den Anspruch der theoretischen Fundiertheit in der
Psychologie als auch deren funktionale Korrektheit mit Hilfe realer Experimentdaten nachweisen
können.
